\documentclass[11pt,a4paper]{jsarticle}
%

\usepackage{amsmath,amssymb}
\usepackage{bm}
\usepackage[dvipdfmx]{color}
\usepackage[dvipdfmx]{graphicx}
\usepackage{ascmac}
\usepackage{listings, jlisting}
\lstset{%
  basicstyle={\small},%
  identifierstyle={\small},%
  commentstyle={\small\itshape},%
  keywordstyle={\small\bfseries},%
  ndkeywordstyle={\small},%
  stringstyle={\small\ttfamily},
  frame={tb},
  breaklines=true,
  columns=[l]{fullflexible},%
  numbers=none,%
  xrightmargin=0zw,%
  xleftmargin=3zw,%
  numberstyle={\scriptsize},%
  stepnumber=1,
  numbersep=1zw,%
  lineskip=-0.5ex%
}


%
\setlength{\textwidth}{\fullwidth}
\setlength{\textheight}{40\baselineskip}
\addtolength{\textheight}{\topskip}
\setlength{\voffset}{-0.2in}
\setlength{\topmargin}{0pt}
\setlength{\headheight}{0pt}
\setlength{\headsep}{0pt}

%
\newcommand{\divergence}{\mathrm{div}\,}  %ダイバージェンス
\newcommand{\grad}{\mathrm{grad}\,}  %グラディエント
\newcommand{\rot}{\mathrm{rot}\,}  %ローテーション
%
\title{チャレンジサイト・メカニックカモノハシ2019\\マイクロマウスシミュレータExercise}
\author{ER17045 立道壱太郎}
\date{\today}
\begin{document}
\maketitle
%
%
\section{この資料について}
メカニックカモノハシではマイクロマウス大会に参加することでメンバーの技術向上を図ります。


\section{マイクロマウスパッケージの導入}

\subsection{workspaceの作成}
メカニックカモノハシ用のworkspaceを新たに作りましょう。

\begin{lstlisting}[frame=single, caption=workspaceの作成, label=create_workspace]
mkdir -p ~/mp_ws/src
cd ~/mp_ws/src/
catkin_init_workspace
echo "source ~/mp_ws/devel/setup.bash" >> ~/.bashrc
\end{lstlisting}


\subsection{マイクロマウスパッケージの導入}
githubから、パッケージをcloneしてcatkin\_makeします。
\begin{lstlisting}[frame=single, caption=catkin\_make, label=catkin_make]
cd ~/mp_ws/src
git clone 
cd ~/mp_ws
catkin_ws
source ~/.bashrc
\end{lstlisting}

以下のコマンドを入力し、ディレクトリを移動できれば成功です。
\begin{lstlisting}[frame=single, caption=roscd, label=roscd]
roscd micro_mouse
\end{lstlisting}




\newpage

\section{関連パッケージの導入}
マイクロマウスパッケージの動作に必要な関連パッケージを導入します。
以下のコマンドを実行してください。(全部で一行)
\begin{lstlisting}[frame=single, caption=roscd, label=roscd]
sudo apt install ros-kinetic-turtlebot ros-kinetic-turtlebot-msgs ros-kinetic-turtlebot-teleop 
\end{lstlisting}

\section{Exercise}
\subsection{Pythonをマスターしよう}


\newpage
\subsection{マウスを動かそう}

\lstinputlisting[caption=micro\_mouse/script/chapter/1/moving\_mouse.py,label=moveing_mouse,numbers=left]{./../../script/chapters/1/moving_mouse.py}


\newpage
\lstinputlisting[caption=micro\_mouse/script/chapter/1/going\_mouse.py,label=going_mouse,numbers=left]{./../../script/chapters/1/moving_mouse.py}


\newpage
\lstinputlisting[caption=micro\_mouse/script/chapter/1/sensing\_mouse.py,label=sensing_mouse,numbers=left]{./../../script/chapters/1/moving_mouse.py}




\newpage
\subsection{wallPublisherを使いこなそう}




\newpage
\subsection{pathPublisherを使いこなそう}




\newpage
\subsection{左手法・拡張左手法・足立法を理解しよう}

\lstinputlisting[caption=micro\_mouse/script/chapter/3/left\_hund.py,label=left_hund,numbers=left]{./../../script/chapters/3/left_hund.py}


\newpage
\lstinputlisting[caption=micro\_mouse/script/chapter/3/left\_hund\_ex.py,label=left_hund_ex,numbers=left]{./../../script/chapters/3/left_hund_ex.py}




\newpage
\subsection{経路計画を理解しよう}



%\lstinputlisting[caption=moving\_mouse.py,label=moveing_mouse,numbers=left]{./../script/chapters/1/moving_mouse.py}


\begin{thebibliography}{99}
\bibitem{install_ubuntu16} Windows10とUbuntu16.04のデュアルブート環境構築\\https://qiita.com/medalotte/items/4bb5cfa709e93d044f1c
\bibitem{install_ubuntu18} Windows10とUbuntu18.04をデュアルブートする.\\https://qiita.com/yo\_kanyukari/items/2a944a300db22482c696
\end{thebibliography}%
%
\end{document}
